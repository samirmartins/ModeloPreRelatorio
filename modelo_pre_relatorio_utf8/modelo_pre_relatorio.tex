\documentclass[12pt,a4paper]{article}
\usepackage[utf8]{inputenc}
\usepackage[T1]{fontenc}
\usepackage[brazil]{babel}
\usepackage{ae}
\usepackage{amsmath,amsfonts,amssymb}
\usepackage{color}
\usepackage{graphicx,xr}
\usepackage{graphics}
\usepackage{tabularx}
\usepackage{times}
\usepackage{url}
\usepackage{natbib} 
\usepackage[pdftex,left=2cm,top=2cm,right=2cm]{geometry}
\usepackage{fancyhdr,epsfig,psfrag}
\usepackage{indentfirst}
\usepackage{float}
\renewcommand{\thetable}{\Roman{table}}
\renewcommand{\baselinestretch}{1.5} 





\begin{document}
\newtheorem{obs}{Observação}
\newtheorem{etp}{Etapa}
\newtheorem{ext}{Extra}
\selectlanguage{brazil}
\author{Samir Angelo Milani Martins}


\date{São João del-Rei, \today}

\title{Trabalho final em grupo - Instrumentação e Medidas}
\pagestyle{fancy}

\lfoot{} \rfoot{ \hfill \small \thepage/\pageref{lastpage}} 
\cfoot{} \chead{ }
\rhead{\small Modelo de Pré-relatório - Laboratório de Circuitos I} 
\lhead{\small }


  \begin{minipage}{\linewidth}
      \centering
      \begin{minipage}{0.25\linewidth}
      \begin{figure}[H]
         \centering
	   \includegraphics[angle=0, scale=.3]{ufsj_logo.jpg}\\       	
          \end{figure}
      \end{minipage}
      \hspace{0.05\linewidth}
      \begin{minipage}{0.45\linewidth}
       	\begin{eqnarray}
	  \begin{array}{c}
	 \mbox{Universidade Federal de São João del-Rei - UFSJ} \nonumber \\
	 \mbox{Departamento de Engenharia Elétrica - DEPEL} \nonumber \\
	  \mbox{Prof. Samir Angelo Milani Martins} \nonumber \\
	 \mbox{São João del-Rei, \today}
	 \end{array}
	 \end{eqnarray}
      \end{minipage}
  \end{minipage}

\vspace{1cm}

\begin{center}
\section*{Modelo de Pré-Relatório de Laboratório de Circuitos I} 

\subsection*{Título da prática}

\end{center}

\begin{flushleft}

\textbf{Aluno:} \\

\textbf{Matrícula: }


\end{flushleft}

\section*{Resumo}

Apresentem aqui as principais ideias e os principais resultados obtidos. Esta seção precisa ser bem escrita e apresentada, pois pode definir se o leitor irá continuar ou não a leitura.

\section{Introdução}

Seção responsável por apresentar ao leitor aspectos iniciais do tema a ser abordado. 

\section{Embasamento Teórico}

Todo o embasamento teórico necessário para execução do experimento deve ser descrito e detalhado nesta seção.

\section{Cálculos Utilizados}

Nesta seção o aluno deve apresentar o memorial de cálculo utilizado no contexto da prática.

\section{Resultados Computacionais e Discussões}

Todos os resultados devem ser apresentados aqui. Utilizem tabelas, figuras e texto para isto. Após a apresentação de resultados, é necessário discutí-los. 

\section{Considerações Finais}

Nesta seção deve-se fazer uma retomada dos principais aspectos apresentados no trabalho, destacando as principais dificuldades encontradas e contribuições da prática. 

\section{Referências Bibliográficas}

Referências bibliográficas utilizadas. Devem ser inseridas de acordo com as normas ABNT.


 
\label{lastpage}
\end{document}